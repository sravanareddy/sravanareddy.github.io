% LaTeX file for resume 
% This file uses the resume document class (res.cls)

\documentclass{res} 
% the margin option causes section titles to appear to the left of body text 
\textwidth=6.1in % increase textwidth to get smaller right margin
\usepackage{hyperref,url}
\usepackage{enumitem}

\hypersetup{
  colorlinks = true,
  urlcolor=blue,
  pdfauthor = {\name Sravana Reddy},
  pdfkeywords = {computer science, computational linguistics, linguistics, natural language
  processing, speech recognition, machine learning, computational biology, data science, data analysis},
  pdftitle = {Sravana Reddy},
  pdfsubject = {Resume},
  pdfpagemode = UseNone
}

\begin{document} 
 
\name{ 
SRAVANA REDDY\\[12pt]} % the \\[12pt] adds a blank line after name
 
\address{HB 6220, Dartmouth College \\ Hanover, NH 03755\\ 781-985-8237   }
\address{\href{mailto:sravana@cs.dartmouth.edu}{sravana@cs.dartmouth.edu}\\
\url{http://www.cs.dartmouth.edu/~sravana}}


 
\begin{resume} 

\section{Current Position}

Neukom Postdoctoral Fellow, {\bf Dartmouth College}, Hanover, NH. \hfill Sep 2012 - present
\begin{itemize}[noitemsep]
\item Fellowship for interdisciplinary computational research
\item Researched linguistic patterns on social media by mining large amounts of data
\item Built a  research system and a web application (\href{http://darla.dartmouth.edu}{darla.dartmouth.edu}) for  automated dialect  analysis of speech  
\item Research resulted in publications and talks, including invited papers, tutorials, and press articles
\item Advised 10 student research, thesis, and software projects
\end{itemize}

\section{Education}

Ph.D. in Computer Science, {\bf The University of Chicago}, Chicago, IL. \hfill Aug 2012 
\begin{itemize}[noitemsep]
\item Thesis: {\em Learning Pronunciations from Unlabeled Evidence}
\item Extensive experience as a research assistant on various NLP and machine learning projects
\item Nine publications on unsupervised machine learning for speech and text 
\end{itemize}

M.S. in Computer Science, {\bf The University of Chicago}, Chicago, IL. \hfill Aug 2009 
\begin{itemize}[noitemsep]
\item Thesis: {\em Part of Speech Induction Using Non-Negative Matrix Factorization}
\item McCormick Fellowship
\end{itemize}

B.S. in Computer Science, Mathematics, Creative Writing, {\bf Brandeis University}, \\Waltham, MA. \hfill May 2006
\begin{itemize}[noitemsep]
\item Senior Thesis: {\em Computational Tools for the Analysis of Poetry}
\item Wien Scholarship; Jerome Schiff Fellowship; Andrew Grossbardt Memorial Prize; Highest Honors
\end{itemize}

\section{Programming Languages}

Python, MATLAB, R, bash, JavaScript, C, C++, SQL, Scheme, Java, Perl, server-side programming

%{\bf Toolkits and Programs}: HTK, CMUSphinx, scikit-learn, SciPy, Praat, \LaTeX, PBS, Apache


\section{Research Experience}
Intern, {\bf University of Southern California} ISI, Marina del Rey, CA.  \hfill Jun - Aug 2010 \& 2011 
\begin{itemize}[noitemsep]
\item NLP research projects on the decipherment of  manuscripts and codes,
and analysis of creative and literary language (\url{https://github.com/sravanareddy})
\item Resulted in three publications in leading conferences, and invited talks
\end{itemize}

Intern, {\bf Mitsubishi Electric Research Laboratories} (MERL), Cambridge, MA.  \hfill  Jul - Sep 2009
\begin{itemize}[noitemsep]
\item Developed techniques for pronunciation modeling; publication in a leading speech conference
\end{itemize}

Intern, Perseus Project, {\bf Tufts University}, Medford, MA.\hfill Jul - Sep 2006
\begin{itemize}[noitemsep]
\item Configured and tested a program for OCR; presented results at a digital humanities conference
\end{itemize}

Intern, Robotics Institute, {\bf Carnegie Mellon University}, Pittsburgh, PA. \hfill Jun - Aug 2005
\begin{itemize}[noitemsep]
\item Implemented algorithms for visual texture recognition 
\end{itemize}


\section{Teaching Experience}
 
 Instructor, {\bf Linguistic Society of America Summer Institute}, Chicago, IL. \hfill Jul 2015
 \begin{itemize}[noitemsep]
 \item Selected to teach an intensive two-week course on ``big data'' statistical analysis of language  
 \end{itemize}
 
Instructor, \href{http://www.cs.dartmouth.edu/~sravana/classes/f14compling/}{Computational Linguistics}, {\bf Dartmouth College}\hfill 2013-2014
\begin{itemize}[noitemsep]
\item Taught 3 courses on NLP and machine learning to 100 computer science and linguistics students
\item Prepared curriculum from scratch, designed assignments, activities, and teaching tools, held extensive help sessions, reviewed mathematical fundamentals, advised students on final projects
\item Teaching materials subsequently used by instructors in other universities
\end{itemize}

Lab Instructor and Teaching Assistant, 13 CS courses, {\bf The University of Chicago} \hfill 2009-2011
 \begin{itemize}[noitemsep]
\item Designed labs, instructed hands-on programming sessions, held office hours, graded homework
\end{itemize}

Community Outreach  \hfill 2012-2015
 \begin{itemize}[noitemsep]
\item Local organizer of the North American Computational Linguistics Olympiad (\href{http://www.cs.dartmouth.edu/~naclo/}{NACLO}) 
\item Led problem-solving and information sessions at Hanover High School
\end{itemize}

\section{Publications and Peer-Reviewed Conference Talks}

\begin{itemize}[noitemsep]
\item Sravana Reddy and James Stanford. A Web Application for Automated Dialect Analysis. {\em Under review at} NAACL. 2015.

\item Sravana Reddy and James Stanford. Is the Future Almost Here? Large-Scale Completely Automated Vowel Extraction from Free Speech. {\em New Ways of Analyzing Variation} (NWAV). 2014. 
Invited to a special issue of {\em Penn Working Papers in Linguistics} and {\em Linguistics Vanguard}, forthcoming in 2015. 

\item Sravana Reddy, Joy Zhong, and James Stanford. A Twitter-Based Study of Newly Formed Clippings in American English. {\em Annual Meeting of the American Dialect Society} (ADS). 2014.

\item Sravana Reddy and Kevin Knight. Decoding Running Key Ciphers. In {\itshape Proceedings of the Annual Meeting of the Association for Computational Linguistics} (ACL). 2012.

\item Sonjia Waxmonsky and Sravana Reddy. G2P Conversion of Proper Names Using Word Origin Information. In {\itshape Proceedings of the Annual Conference of the North American Chapter of the Association for Computational Linguistics} (NAACL). 2012.

\item Sravana Reddy and Evando Gouv\^ea. Learning from Mistakes: Expanding Pronunciation Lexicons Using Word Recognition Errors. In {\itshape Proceedings of the Annual Conference of the International Speech Communication Association} (INTERSPEECH). 2011.

\item Sravana Reddy and Kevin Knight. Unsupervised Discovery of Rhyme Schemes. In {\itshape Proceedings of the Annual Meeting of the Association for Computational Linguistics} (ACL). 2011.

\item Sravana Reddy and Kevin Knight. What we know about the Voynich Manuscript. In {\itshape Proceedings of the ACL LaTeCH Workshop}. 2011.

\item Sravana Reddy and John Goldsmith  An MDL-based approach to extracting subword units for grapheme-to-phoneme conversion. In {\itshape Proceedings of the Annual Conference of the North American Chapter of the Association for Computational Linguistics} (NAACL). 2010.

\item Sravana Reddy and Sonjia Waxmonsky.  Substring-based Transliteration with Conditional Random Fields. In {\itshape Proceedings of the ACL Named Entities Workshop}. 2010.

\item Sravana Reddy. Understanding Eggcorns. In {\itshape Proceedings of the NAACL Workshop on Computational Approaches to Linguistic Creativity}. 2010.

\item Sravana Reddy and Gregory Crane. A Document Recognition System for Early Modern Latin. {\em Chicago Colloquium on Digital Humanities and Computer Science}. 2006.

\end{itemize}

%\section{Nationality}
%Indian citizen. H-1B visa.

\end{resume} 
\end{document} 



